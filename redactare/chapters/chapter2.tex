\chapter{Solution proposed}

\section{Implementation overview}

The implementation of the CNF conversion algorithm is done using three main methods:
\texttt{applyAtTop}, \texttt{applyRule}, and \texttt{convertToCNF}. The code snippets 
that follow contain the full specifications, preconditions and postconditions, but the
implementation is partially skipped 
The \texttt{applyAtTop} receives a formula f. If f can be changed using one of
the nine equivalences then it returns the new version, if not, the initial formula
is returned. The \texttt{measure} function return a tuple that is used as the variant
for the termination condition.

\begin{lstlisting}[language=Scala, caption={Method \texttt{applyAtTop}}]
def applyAtTop(f : FormulaT, orsAboveLeft : BigInt, 
        andsAboveLeft : BigInt) : FormulaT = {
require (orsAboveLeft >= 0)
require (andsAboveLeft >= 0)
require (validFormulaT(f))
f match {
    case DImplies(f1, f2) => applyRule1(f, orsAboveLeft, andsAboveLeft)
    case Implies(f1, f2)  => applyRule2(f, orsAboveLeft, andsAboveLeft)
    case Or(f1, f2)       => {        
        if(f2.isInstanceOf[And]) {
            applyRule3(f, orsAboveLeft, andsAboveLeft)}
        [...]}
    [...]}
    [...]
} ensuring(res =>
(validFormulaT(res))
    && (if (f == res) !f.isInstanceOf[Implies] else true)
    && (if (f == res) !f.isInstanceOf[DImplies] else true)
    && (f == res || smaller(measure(res, orsAboveLeft, andsAboveLeft),
    measure(f, orsAboveLeft, andsAboveLeft))))
\end{lstlisting}


We also define 9 methods: \texttt{applyRule1, applyRule2, ... applyRule9} 
which apply the coresponding rule given the assurance that the input formula
has a specific format. For example, \texttt{applyRule1} will require a formula
where the root node is a double implication.

\begin{lstlisting}[language=Scala, caption={Method \texttt{applyRule1}}]
def applyRule1(f : FormulaT, orsAboveLeft : BigInt, 
        andsAboveLeft : BigInt) : FormulaT = {
    require(formula.validFormulaT(f))
    require(f.isInstanceOf[DImplies])
    require(orsAboveLeft >= 0)
    require(andsAboveLeft >= 0)
    val DImplies(f1, f2) = f
    val result = And(Implies(f1, f2), Implies(f2, f1))
    [...] // assertions necessary for verification
    result
} ensuring(res =>
    (validFormulaT(res))
    && (weightOfAnds(res) <= weightOfAnds(f))
    && (countDImplies(res) < countDImplies(f))
    && smaller(measure(res, orsAboveLeft, andsAboveLeft),
        measure(f, orsAboveLeft, andsAboveLeft)))           
\end{lstlisting}

The method \texttt{applyRule} a formula and propagates the \texttt{applyAtTop} method
down the tree for \texttt{And, Or} and \texttt{Not} nodes. Calling this method will
trigger only one rule application.

\begin{lstlisting}[language=Scala, caption={Method \texttt{applyRule}}]
def applyRule(f : FormulaT, orsAboveLeft : BigInt, 
        andsAboveLeft : BigInt) : FormulaT = {
    require (orsAboveLeft >= 0)
    require (andsAboveLeft >= 0)
    require (validFormulaT(f))
    val res : FormulaT = applyAtTop(f, orsAboveLeft, andsAboveLeft)
    if (res != f) {
        res // no further changes needed
    } else if (f.isInstanceOf[Or]) {
        val Or(f1, f2) = f // destruct f knowing the root node
        val f1_step = applyRule(f1, orsAboveLeft, andsAboveLeft)
        f1_step
        [...]} // treat the rest of the cases with Or root
    [...] // treat the rest of the cases
}ensuring(res =>
    (validFormulaT(res))
    && (f == res || smaller(measure(res, orsAboveLeft, andsAboveLeft),
        measure(f, orsAboveLeft, andsAboveLeft))))
\end{lstlisting}

\texttt{convertToCNF} is the entry method in the algorithm. It calls \texttt{applyRule}
in a loop until there are no changes made to the formula. The \texttt{decreases} annotation
for this method is a 5-uple of variants that represent the termination measure of the implementation.
The termination method will not be further explained as it is already covered as part
of the bachelor's thesis\cite{licenta}.

\begin{lstlisting}[language=Scala, caption={Method \texttt{applyRule}}]
def convertToCNF(f : FormulaT) : FormulaT = {
    decreases (weightOfAnds(f), countDImplies(f), 
        countImplies(f), countOrPairs(f, 0), countAndPairs(f, 0))
    require (validFormulaT(f))

    val res = applyRule(f, 0, 0)
    if(res != f) {
        convertToCNF(res)
    } else {
        res
    }
}
\end{lstlisting}


\section{Comparison between Dafny and Stainless}

In this section we will discuss the main differences between the two tools. The usual differences
like syntax, and dependencies will not be treated as we focus on the specific elements that influence
the formal verification development flow.

\subsection{Performance}

\emph{All the tests are performed on Stainless version 0.9.4 using a standard laptop: AMD Ryzen 7 4700U (2.0 GHz), 16GB RAM and Ubuntu 16.04.7 LTS.}

From the start we notice that code compilation in scala is slower. This is a known factor that one 
should expect, even without using Stainless. A first factor would be that the program 
is ran on a JVM so start up times are slow, up to 2-4 seconds. If the setup is correct, this will
be mitigated by the compiler being kept in a running JVM. SBT as well as IDEs can handle this.
The authors of Stainless already provide a sbt template project in our aide. \cite{sbt}

In order to improve the performance, Stainless uses a persistent cache mechanism to speed up
verification times. By doing this, the only solution for the developer is to write the program
in an incremental way, and compile the solution after each piece of the proof. While
breaking down the proof into smaller methods is a general clean code strategy that adds a boost 
of performance in Dafny, it is \emph{mandatory}
when writing non trivial programs in Stainless. Even if it wasn't the case in for this algorithm, one
can expect the need to refactor bigger proofs into smaller predicates and lemmas when porting a project.

The method which pushes the verification engine to the edge is \texttt{applyRule} where the verification time
is bigger than expected without the proper verification: the compilation crashed over 10 minutes for the whole method. 
Initially, the solution was to perform the verification for each of the rules and verify partial proofs in
several sessions. While this approach is not the most elegant one, it is business as usual when 
writing a proof for the first time. After some improvements, Stainless manages to compile the method without the
workaround mentioned above.

Using Stainless, the verification times is under one second for common proofs like \texttt{Rule5Prop} 
and a couple of seconds when infering and using their results in more complex ones like \texttt{applyAtTop}.
This means that when trying to compile from ground up the whole project the duration will reach 2 minutes.
Meanwhile, Dafny will perform the same verifications in under 1 minute\cite{wollic}.

\newpage

\subsection{Induction}

In order to verify a program written in Scala, it has to be implemented using a purely
functional subset of the language, which the authors of Stainless call \emph{Pure Scala}\cite{pure_scala}.

One of the top-level declarations supported is an Algebraic Data Type (ADT)\cite{pure_scala}

We define our data type for a formula as follows:

\begin{lstlisting}[language=Scala, caption={ \texttt{FormulaT} data type}]
sealed abstract class FormulaT
case class Var(v: BigInt) extends FormulaT
case class Not(f1: FormulaT) extends FormulaT
case class Or(f1: FormulaT, f2: FormulaT) extends FormulaT
case class And(f1: FormulaT, f2: FormulaT) extends FormulaT
case class Implies(f1: FormulaT, f2: FormulaT) extends FormulaT
case class DImplies(f1: FormulaT, f2: FormulaT) extends FormulaT
\end{lstlisting}

We notice methods where the verification can not succeed if we do not specify the usage of 
the \emph{induct} tactic by using \texttt{@induct}.

\begin{lstlisting}[language=Scala, caption={ Lemmas with mandatory induction}]
@induct
def variablesUpToMonotone(f : FormulaT, 
        n1 : BigInt, n2 : BigInt) : Boolean = {
    require(variablesUpTo(f, n1) && (n1 <= n2))
    variablesUpTo(f, n2)
}.holds

@induct
def mult2_upper(x : BigInt) : Boolean = {

    require(x >= 0)
    2 * x < 1+ pow(2, x)
}.holds
\end{lstlisting}

During the implementation I use this feature only when is mandatory as it 
can slow down the verification time or even invalidate the proof. 
For example,by addnotating two measure functions to use the induction tactic 
I notice that if the behaviour is too complex the verification will fail.
In the following listing are two recursive functions that call only themselves and 
\texttt{pow}. The verification will fail only for \texttt{weightOfAnds} when trying to use induction.

\begin{lstlisting}[language=Scala, caption={ Lemmas with mandatory induction}]
@induct // verification fails
def weightOfAnds(f : FormulaT) : BigInt = {
f match {
  case Var(v)     => BigInt(2)
  case Not(f1)    => pow(2, weightOfAnds(f1))
  case Or(f1, f2) => weightOfAnds(f1) * weightOfAnds(f2)
  case And(f1, f2) => weightOfAnds(f1) + weightOfAnds(f2) + 1
  case Implies(f1, f2)  => pow(2, weightOfAnds(f1)) * weightOfAnds(f2)
  case DImplies(f1, f2) => 
      pow(BigInt(2), weightOfAnds(f1)) * weightOfAnds(f2) +
      pow(BigInt(2), weightOfAnds(f2)) * weightOfAnds(f1) + 1
}
} ensuring(
    res => res >= 2)

@induct // verification passes
def countDImplies(f: FormulaT) : BigInt = {
f match {
  case Var(v)           => BigInt(0)
  case Not(f1)          => countDImplies(f1)
  case Or(f1, f2)       => countDImplies(f1) + countDImplies(f2)
  case And(f1, f2)      => countDImplies(f1) + countDImplies(f2)
  case Implies(f1, f2)  => countDImplies(f1) + countDImplies(f2)
  case DImplies(f1, f2) => 1 + pow(2, countDImplies(f1) + countDImplies(f2))
}
} ensuring(
    res => res >= 0)
\end{lstlisting}


\subsection{Proof autonomy levels}

Both tools show specific levels of autonomy depending on the context in which the verification
process takes place. For example, Dafny is able to prove simple properties of the power function
over integers without our help while Stainless requires input.

\begin{lstlisting}[language=Scala, caption={ Lemmas with mandatory induction}]
def pow_increases(a : BigInt) : Boolean = {
    require(a >= 1)
    (a < pow(2, a)) because {
        if (a == 1) true
        else        pow_increases(a - 1)
    }
}.holds

def pow_monotone(a : BigInt, b : BigInt) : Boolean = {
    require(0 <= a)
    require(a <= b)
    (pow(2, a) <= pow(2, b)) because {
        if (a == b)  pow(2, a) == pow(2, b) 
        else         pow_monotone(a, b - 1)
    }
}.holds
\end{lstlisting}

The \texttt{pow_increases} lemma states that $\forall a \in \mathbb{N}, a < 2 ^{a}$ while \texttt{pow_monotone}
states that $\forall a, b \in \mathbb{N}, a \leq b  \Rightarrow 2 ^{a} \leq 2 ^{b}$.

Another example of function, that has a \texttt{FormulaT} type parameter this time, is \texttt{Rule5PropAux}.
This lemma states that the measure function \texttt{countOrPairs} 
evaluation monotonically increases when the parameter \texttt{orsAboveLeft} is incremented.
While this is proven by default in Dafny, here we have to manually tell the verification engine 
to make a recursive verification by calling the lemma over the child nodes of the formula.


Based on the examples above, Stainless seems to lack autonomy for \emph{simple} inductive proofs.
When it comes to demonstrations where we use a chain of assertions the level of autonomy is higher 
than in Dafny.

\begin{lstlisting}[language=Scala, caption={Lemma \texttt{Rule5Prop}}]
// verification in dafny
lemma Rule5Prop(f1 : FormulaT, f2 : FormulaT, f3 : FormulaT,
    orsAboveLeft : int)
    requires orsAboveLeft >= 0;
    ensures countOrPairs(Or(Or(f1, f2), f3), orsAboveLeft) <
        countOrPairs(Or(f1, Or(f2, f3)), orsAboveLeft);
{
    assert countOrPairs(Or(f1, Or(f2, f3)), orsAboveLeft) ==
        countOrPairs(f1, orsAboveLeft) + countOrPairs(f2, orsAboveLeft + 1) +
        countOrPairs(f3, orsAboveLeft + 2) + orsAboveLeft + orsAboveLeft + 1;
    assert countOrPairs(Or(Or(f1, f2), f3), orsAboveLeft) ==
        countOrPairs(f1, orsAboveLeft) +
        countOrPairs(f2, orsAboveLeft + 1) +
        countOrPairs(f3, orsAboveLeft + 1) + orsAboveLeft + orsAboveLeft;
    Rule5PropAux(f3, orsAboveLeft + 1);
    assert countOrPairs(f3, orsAboveLeft + 1) 
            <= countOrPairs(f3, orsAboveLeft + 2); 
    assert countOrPairs(Or(Or(f1, f2), f3), orsAboveLeft) <
        countOrPairs(Or(f1, Or(f2, f3)), orsAboveLeft);
}
\end{lstlisting}

The lema above is extracted from \emph{Verifying the Conversion into CNF in Dafny} \cite{wollic}.
Here, Stainless needs only the call to \texttt{Rule5PropAux}
and the last assertion in order to obtain the same proof.