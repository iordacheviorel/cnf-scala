\chapter{Description of the problem}

\section{The transformation into Conjunctive Normal Form}

The boolean satisfiabily problem, or SAT, is the problem of deciding if 
a formula from the boolean logic is satisfiable. A formula is satisfiable when
there exists at least one assignment of boolean values over the variables
such as the evaluation of the formula will be true. If there is no assignment
to satisfy this condition then we say that the formula is unsatisfiable.

The SAT problem is seen most of the times as CNF-SAT which adds the contition
that the formula has to in conjunctive normal form. Because the probability
of expressing a problem using directly logical formulas in the CNF format we 
will need to do this conversion before moving forward.

In order to convert a formula into conjunctive normal form, the nine following equivalences have to be used:

\begin{enumerate}
  \item $( \varphi_1 \Leftrightarrow \varphi_2) \equiv ((\varphi_1 \Rightarrow \varphi_2) \land (\varphi_2 \Rightarrow \varphi_1));$
  \item $(\varphi_2 \Rightarrow \varphi_1) \equiv (\neg \varphi_1 \lor \varphi_2)$
  \item $(\varphi_1 \lor (\varphi_2 \land \varphi_3)) \equiv ((\varphi_1 \lor \varphi_2) \land (\varphi_1 \lor \varphi_3));$
  \item $((\varphi_2 \land \varphi_3) \lor \varphi_1) \equiv ((\varphi_2 \lor \varphi_1) \land (\varphi_3 \lor \varphi_1));$
  \item $(\varphi_1 \lor (\varphi_2 \lor \varphi_3)) \equiv ((\varphi_1 \lor \varphi_2) \lor \varphi_3));$
  \item $(\varphi_1 \land (\varphi_2 \land \varphi_3)) \equiv ((\varphi_1 \land \varphi_2) \land \varphi_3));$
  \item $\neg (\varphi_1 \lor \varphi_2) \equiv (\neg \varphi_1 \land \neg \varphi_2);$
  \item $\neg (\varphi_1 \land \varphi_2) \equiv (\neg \varphi_1 \lor \neg \varphi_2);$
  \item $\neg \neg \varphi \equiv \varphi;$
\end{enumerate}

The first two equivalences will assure that all simple and double implications will be removed from the formula.

The equivalences 3 and 4 assure us that all the disjunctions will be brought on a lower level than the conjunctions.

The equivalences 5 and 6 assure that we can associate the parantheses from left to write so that we can write 
$\varphi_1 \lor \varphi_2 \lor  . . . \lor \varphi_n$ instead of
$(((\varphi_1 \lor \varphi_2) \lor  . . .) \lor \varphi_n)$. 

The equivalences 7 and 8, or the De Morgan's laws, will push the negations to the leaves of the tree,
while the last equivalence will remove the double negations that may appear.

\section{Formal verification tools}

In order to implement a solution for the problem described above we have to choose a language which
offers suppport for formal verification. My first choice was to use Dafny which combines both functional
and object oriented programing paradigms. It suppoerts formal verification through adnotations of properties
using preconditions and postconditions and the usage of loop invariants in order to formulate termination conditions.
Dafny also offers support for common concepts such as classes and trait inheritance or inductive datatypes which I will
use in order to represent logical formulas in the implementation that follows.

While Dafny can compile the code into several popular programing languages I wanted to lean technology wise
towards a language from the JVM ecosystem. I choose to you scala, as it has the support of the verification tool 
Stainless, developed by LARA at EPFL. \cite{stainless_home}

We can consider Stainless as a way to use Scala in order to ensure functional corctness during the development stage.
The authors tend to compare Stainless to proof assistants such as Isabelle/HOL, Coq, or ACL2 even if initially it was meant
to be a program verifier, much closer to Dafny. \cite{stainless_home}

Stainless uses Inox, a solver for higher-order function programs, which relies SMT solvers such as Z3, CVC4 and Princess.
In a similar approach, Dafny builds on the Boogie intermediate verification language\cite{boogie_docs}
which is aided by the same Z3 theorem prover.


