\chapter*{Introduction} 
\addcontentsline{toc}{chapter}{Introduction}

% aceasta va conține motivația alegerii temei, gradul de noutate a temei,
% obiectivele generale ale lucrării, metodologia folosită, descrierea sumară a soluției,
% structura lucrării (titlul capitolelor și legătura dintre ele). Introducerea nu se
% numerotează ca celelalte capitole.

%introducere
The thesis presents the differences between Dafny and Stainless using the
implementation of the textbook conversion into CNF for propositional logic formulaes as an example.
The two implemenatations of the algorithm are fully verified meaning that they provide
proof of termination which is machine-checked and functional corectness.

By functional corectness we understand the degree to witch our implementation provides
the correct or expected results. We may want to have this assurance because programs
perform the actions they are \textit{told} to, not the ones we intend them to.
In industry, the corectness of programs is checked, partially, by testing. Even so, there
are instances where the functional corectness of a complex task can not be proven or satisfied
by conventional means.

%motivația alegerii temei
The reason for this comparison between formal verification tools arises from
the fact that while being similar at a first glance they have their own peculiarities
which may help us choose one over the other as the better candidate for a specific problem.


%gradul de noutate a temei


%obiectivele generale ale lucrării
The objective of this work is to provide a comparison by highlighting the advantages 
and disadvantages of both tools when using them on a practical example. This paralel may
come as useful to programmers who knows one of the languages and wants to explore the other, 
wants to choose between them for their implemenation or other students who are learning  
about formal verification.
