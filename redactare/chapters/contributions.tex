\chapter*{Contributions} 
\addcontentsline{toc}{chapter}{Contributions}

\emph{First contribution.} We verify that an algorithm based
on applying the nine equivalences above is functionally correct. The
most difficult part is to prove termination, for which we use a
carefully designed 5-tuple as a variant. To our knowledge, this is
incidentally the first proof (paper or computer-checked) of
termination of the nine rules. Indeed, in all logic textbooks that we
surveyed, termination is only proved for a certain strategy (first
applying rules 1 and 2, then finding a NNF using rules 7-9, etc.) of
applying the rules. The interest into this is of theoretical interest,
since other strategies (such as bringing the formula into NNF first)
are easier to prove.

\emph{Second contribution}. We verify an implementation of Tseitin’s
transformation in Dafny. The main difficulty is to find the right
inductive invariant. There are also some technical difficulties with
the verification: our implementation pushes the prover to its limits
and requires carefully designed lemmas, helper predicates, and
assertions in order to verify successfully. For this approach,
termination is established by Dafny automatically, since the function
is recursive on the formula ADT. To our knowledge, this is the first
auto-active proof of Tseitin's transformation.

\emph{Third contribution}. We verify the first algorithm both in Dafny and Scala (using the 
Stainless tool). By doing this, we create a comparison between 
the two languages highlighting the advantages and disadvantages of 
each other over a complex example.